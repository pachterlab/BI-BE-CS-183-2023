\documentclass[11pt]{exam}
\usepackage{amsfonts,amsthm,amsmath,amssymb,mathrsfs,bbm,dsfont}
\usepackage{hyperref}
\usepackage{nicematrix}
\usepackage{csquotes}\MakeOuterQuote{"}
\qformat{\textbf{Problem \thequestion}\quad (\thepoints)\hfill}
\newcommand{\ind}{\perp\!\!\!\!\perp} 
\newtheorem{theorem}{Theorem}



\begin{document}

\begin{center}

     \textbf{Bi/BE/CS 183 2022-2023\\ Instructor: Lior Pachter\\ TAs: Tara Chari, Meichen Fang, Zitong (Jerry) Wang \vskip 0.15in Problem Set 8}

\end{center}
Submit your solutions as a single PDF file via Canvas by {\bf 8am Tuesday March 7th}. 
\begin{itemize}
  \item If writing up problems by hand, please use a pen and not a pencil, as it is difficult to read scanned submission of pencil work. Typed solutions are preferred.
  \item For problems that require coding, Colab notebooks will be provided. Please copy and save the shared notebook and edit your own copy, which you should then submit by including a clickable link in your submitted homework. Prior to submission make sure that you code runs from beginning to end without any error reports.
  \end{itemize}
  
  
  A hidden Markov model (HMM) has $n$ hidden states and $m$ observed states. At any time $t$, the random variable $X_t$ is the hidden state at time $t$ and the random variable $Y_t$ is the observation at time $t$. There is a $n\times n$ transition matrix $S = \{S_{ij}\}$, where $S_{ij} = \mathbb{P}(X_t = j | X_{t-1} = i)$ is the probability of transitioning from hidden state $i$ to hidden state $j$. There is a $n \times m$ emission matrix $T = \{T_{ij}\}$, where $T_{ij} = \mathbb{P}(Y_t = j | X_t = i)$ is the probability of observing state $j$ while in hidden state $i$.
  
  \begin{questions}


\question[60] Consider the HMM with $n = 2$, $m = 4$ (corresponding to $A$,$C$,$T$,$G$), and an equal of probability of starting in each of the two hidden states. Suppose that

\begin{equation}
S=\left(\begin{array}{cc}
0.8 & 0.2 \\
0.05 & 0.95
\end{array}\right)
\end{equation}
and
\begin{equation}
T=\left(\begin{array}{cccc}
0.2 & 0.5 & 0.1 & 0.2 \\
0.1 & 0.25 & 0.25 & 0.4
\end{array}\right).
\end{equation}

\begin{parts} 
\part[10] Give all sequences of hidden states with non-zero probability of generating the sequence $ACG$ (please provide link to code if not done by hand).
\part[20] Find the most probable sequence of hidden state to have produced the sequence $ACG$ (please provide link to code if not done by hand)
\part[20] Compute $p_{ACG}$, which is the probability of observing the sequence ACG given all possible paths of hidden states (please provide link to code if not done by hand).
\part[5] Compute $p^*_{ACG}$, the probability of observing the sequence $ACG$ via the most probable sequence of hidden states. 
\part[5] For a particular observed DNA sequence, one might consider approximating the probability of observing the sequence by the probability of observing the sequence via the most probable sequence of hidden states. In the case of the sequence $ACG$, would this be a good approximation?
% \part[10] (learning) Suggest a change to the initial probability of starting in one of the two hidden states so as to increase $p_{ACG}$.
\end{parts}

\newpage
% \question[30] Suppose that an HMM with $n=m=2$ (corresponding to observations $0$,$1$) has $p_{000} = p_{111} = 0$. Show that $S$ and $T$ must satisfy the following conditions,

% \begin{equation}
% S=\left[\begin{array}{ll}
% 0 & 1 \\
% 1 & 0
% \end{array}\right], \quad T=\left[\begin{array}{ll}
% 1 & 0 \\
% 0 & 1
% \end{array}\right], \, \text { or } \,\left[\begin{array}{ll}
% 0 & 1 \\
% 1 & 0
% \end{array}\right].
% \end{equation}
% \newpage



\question[40] In this problem you will implement the Needleman-Wunsch algorithm for global alignment of a pair of sequences. Your program will read in a FASTA file containing a pair of DNA sequences, and then run the Needleman-Wunsch algorithm to find their optimal alignment given parameters for matching, mismatching, and unaligned base-pairs. The algorithm constructs an optimal global alignment by dynamic programming.

 \href{https://github.com/pachterlab/BI-BE-CS-183-2023/blob/main/HW8/Problem3.ipynb}{The Problem notebook is here}.

 Your edited version of the notebook \textit{must be submitted } for this problem. Reminder to check that your notebook runs all the way through with the the {\tt Runtime} $\xrightarrow{}$ {\tt Restart} and {\tt Runtime} $\xrightarrow{}$ {\tt Run All} commands.

  \end{questions}



\end{document}
